%! Author = Sten
%! Date = 09-10-2020

\documentclass[11pt]{article}

\usepackage[a4paper,margin=2.5cm]{geometry}

% Packages
\usepackage[fleqn]{amsmath}

\usepackage{amsthm}
\usepackage{amssymb}
\usepackage{stmaryrd}
\usepackage{amsfonts}
\usepackage{etoolbox}
\usepackage{enumerate}
\usepackage{scalerel}
\usepackage{stackengine}
\usepackage{xifthen}
\usepackage{lmodern}
\usepackage{gensymb}
\usepackage{color}
\usepackage{colortbl}
\usepackage{needspace}
\usepackage{enumitem}
\usepackage{mathtools}
\usepackage{bbm}
\usepackage{xargs}
\usepackage{algorithm}
\usepackage{diagbox}
\usepackage{subcaption}
\usepackage[table]{xcolor}
\usepackage[noend]{algpseudocode}
\usepackage{polynom}

\newcommand{\naturals}{\mathbb{N}}
\newcommand{\integers}{\mathbb{Z}}
\newcommand{\reals}{\mathbb{R}}
\newcommand{\complex}{\mathbb{C}}
\newcommand{\rationals}{\mathbb{Q}}

\DeclarePairedDelimiterX{\set}[1]\lbrace\rbrace{\setaux #1||\endsetaux}
\def\setaux#1|#2|#3\endsetaux{\if\relax\detokenize{#2}\relax #1 \else #1 \;\delimsize\vert\; #2 \fi}

% Q.E.D.
\renewcommand{\qedsymbol}{\footnotesize$\boxempty$ \textsc{q.\,e.\,d.}}


\title{\textbf{Generalized VPN project}\\State of affairs}
\author{David Dekker \and Sten Wessel}
\date{\today}

% Document
\begin{document}
    \maketitle
    \hrule


    \section{Introduction}


    \section{Implemented algorithms}
    \begin{itemize}
        \item Enumeration algorithm, finding the optimal hierarchical hubbing solution.

        Recursive, brute-force algorithm, enumerating all possible mappings $h\colon V_T \to V_G$ (with $h(i) = i$ for all terminals $i \in W$).
        Used to check the dynamic programming algorithm, but is too slow to use for reasonably-sized instances.

        \item Dynamic programming algorithm, finding the optimal hierarchical hubbing solution.

        Based on ``Approximability of Robust Network Design'' by Olver and Shepherd, from the proof of Lemma~3.1.
        The dynamic program only finds the optimal solution \emph{value}, and we construct the actual solution (the mapping $h$, and routing template $\set{P_{ij}}$, and bought capacities $u$) from backtracking the DP table.
        In general, the obtained routing template paths may be non-simple, which we `fix' by removing the cycles explicitly.

        \item Semi-infinite MIP formulation, finding an exact solution.

        MIP formulation using row generation.
        Turns out to be slow, as many row generation iterations are needed to find an optimal solution.

        \item Compact MIP formulation, finding an exact solution.

        MIP formulation where the need for row generation is removed by using a clever dualization trick, as described in ``Provisioning Virtual Private Networks under Traffic Uncertainty'' by Alt\i{}n, Amaldi, Belotti and P\i{}nar.

        Using the solution from the DP algorithm, we have attempted to warm start the MIP by providing this solution to the solver.
        However, it seems to be the case this solution is sometimes not used by the solver, but sometimes it speeds up the solving process.
        We have not been able to find (yet) why the solver is not able to do something with the provided solution.

        Both MIP formulations are implemented using Gurobi.
    \end{itemize}


    \section{Experiments}
    \begin{itemize}
        \item Petersen graph
        \begin{itemize}
            \item All ($68\,185$) instances with:
            \begin{itemize}
                \item Unit cost on the edges in $G$,
                \item All possible choices for which nodes in $G$ are a terminal ($\sim 2^{10}$),
                \item All possible \emph{two-union star} trees as the demand tree, with unit capacity on the edges incident to a terminal, and all (relevant) choices for the bridge cost.
                \item For all cases, the LP relaxation has the same objective value as the optimal integer solution.
                In almost all cases, the LP relaxation solution was integral.
            \end{itemize}
            \item ($> 100\,000$) random instances, with:
            \begin{itemize}
                \item Random cost on the edges in $G$,
                \item Random choices for which nodes in $G$ are a terminal,
                \item Randomly built demand tree.
                \item Yielded no interesting results
            \end{itemize}
        \end{itemize}

        \item Augmented Petersen graph
        \begin{itemize}
            \item Adding fifteen groups of terminals
            \begin{itemize}
                \item Consider the Petersen graph where all edges have weight 1.
                \item Add a root node to the Petersen graph and connect it to all vertices with weight 2.
                \item Add a center node and a (terminal) root node to the demand tree and connect them with weight 3, 5 or $\infty$.
                \item For each edge of the original Petersen graph, pick one of the two vertices and consider its two neighbors outside the edge.
                Add three terminals to the demand tree, connect each terminal to a new interior node (weight 1) and connect this new interior node to the center node (weight 1, 2 or 3).
                \item For each setting, we run this experiment approximately 100 times.
                Each iteration took approximately 40 seconds.
                No interesting results so far, but we just realized that the root node was not a terminal in our test case, so we will rerun these.
            \end{itemize}
            \item Adding thirty groups of terminals
            \begin{itemize}
                \item A similar setting, but we now add a group for both endpoints of each edge, instead of picking only one endpoint arbitrarily.
                \item Each test took now approximately two hours, and these also need to run again because the root of the demand tree was not a terminal.
            \end{itemize}

        \end{itemize}
    \end{itemize}


    \section{Generalizing the proof on ring networks}
    If a tree solution is provided, it is possible to define the number of times we need to buy an edge $e$ in the tree.
    This is denoted with $u(e)$ in the paper ``A Short Proof of the VPN Tree Routing Conjecture on Ring Networks'' by Grandoni, Kaibel, Oriolo and Skutella.
    We believe we managed to generalize this definition to the situation where the demand tree is a 2-union star.
    This definition is explained in the slides and in the work-in-progress report ``Generalizing the VPN conjencture''.
    In the report, an attempt to prove Lemma~3 is described, but it was not successful yet.
    Under the current definition of the traffic demands $D^e$, we are not yet able to prove that Claim~1 holds on the bridge (i.e., the edge in the demand tree that connects the two stars).
    So far, other definitions were also not successful.
    It is not clear whether this approach can work, because we are not sure whether a tree solution will actually always exist.

    For the second part of the theorem, we also do not have a proof yet.
    We should note that Olver mentions that he was able to finish a part of the proof in his lecture notes ``Designing networks under uncertain demands (draft)''.
    It is precisely the situation of the generalized VPN network with a 2-union star as demand tree.
    However, he uses a different structure based on vines.
\end{document}
