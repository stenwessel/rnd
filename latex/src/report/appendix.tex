\section{Proving the generalized VPN conjecture on ring networks} \label{app:proof}
When we assume there always exists an optimal hubbing solution for the generalized VPN problem that is also a tree solution, one can try to prove the generalized VPN conjecture with the same ideas as presented in \cite{grandoni2008short}.
However, as stated in Section~\ref{subsec:genvpn:rings}, this assumption can in general not be made.
In any case, our efforts of generalizing the proof under the assumption are layed out in this section.
The ideas here are in no way complete, and some (parts of) proofs are missing for a number of introduced theorems and lemmas.
In particular, Claim~4 from \cite{grandoni2008short} is problematic, and this fails to let the final inductive argument work.

Some background of what steps we are taking is identical to what is described in \cite{grandoni2008short} and we will omit these details.
We focus on showing the differences that are necessary in the proof to adapt to the generalized VPN problem setting.
We use the same notation as used in the aforementioned paper, except where the generalized VPN problem differs from the VPN problem.

For the remainder, we restrict ourselves to instances where $T$ is a \emph{two-union star}, that is, a tree formed by connecting the centers $r_1,\ r_2$ of two star graphs.
We will call this edge $r = \set{r_1, r_2}$ and refer to it as the \emph{bridge} of $T$.

\subsection{Preliminaries}
A number of assumptions are made in \cite{grandoni2008short}, the most important being that we can assume the capacities $b \equiv 1$.
We suspect this does not quite generalize to our case, but we can state the following:

\begin{fact}
    We may assume $b_f = 1$ when $f \neq r$.
\end{fact}
Using the same motivation as in \cite{grandoni2008short}, if the capacity $b_{iu}$ for a terminal $i \in W$ is not unit, construct a new instance with new terminals $i_1, \dots, i_{b_{iu}}$, connected to the site of $i$ in $G$, with edge cost $0$.
Set the capacities of the incident edges in $T$ to unit.
Note that this instance is equivalent, but may change the topology of the graph.

An alternative construction is argued in \cite{grandoni2008short} to keep the topology a ring and we refer to this paper for the details.

\begin{fact}
    We can assume $b$ to be integral.
\end{fact}
This is motivated by scaling by a sufficiently large factor.

\subsection{Pyramidal routing problem}
To goal is to show that there exists an optimal solution to the generalized VPN problem that is a tree solution (or, equivalently, there exists a tree solution to the generalized VPN problem that is optimal).
Consider an optimal tree solution $(\set{P_{ij}}, x)$ to a generalized VPN instance $(G, c, W, T, b)$, with $|W| = k$ and $T$ a two-union star with the unit capacity assumption as above.
Let $\mathcal P_i$ be as in \cite{grandoni2008short} for a fixed $i \in W$.
We define
\[
    \xi(e, \mathcal P_i) \coloneqq \min\set{\alpha(e, \mathcal P_i),\ \beta(e, \mathcal P_i)} + \min\set{n(e, \mathcal P_i) - \alpha(e, \mathcal P_i),\ k - n(e, \mathcal P_i) - \beta(e, \mathcal P_i)}
\]
where
\begin{gather*}
    n(e, \mathcal P_i) \coloneqq |\set{j \in W \setminus \set{i} : e \in P_{ij}}|,\\
    \alpha(e, \mathcal P_i) \coloneqq |\set{j \in W \setminus \set{i} : e \in P_{ij},\ r \in \pi_T(i, j)}|,\\
    \beta(e, \mathcal P_i) \coloneqq |\set{j \in W \setminus \set{i} : e \not\in P_{ij},\ r \not\in \pi_T(i, j)}|,
\end{gather*}
that is, $\alpha(e, \mathcal P_i)$ is the number of paths in $\mathcal P_i$ containing $e$, \emph{while simultaneously the path in the demand tree between $i$ and $j$ crosses the bridge}.
In our setting, we have a different expression for the \emph{required} capacity of edge $e$:
\[
    x(e) = \xi(e, \mathcal P_i) + \min\Big\{b(r),\ n(e, \mathcal P_i) - \xi(e, \mathcal P_i),\ k - n(e, \mathcal P_i) -  \xi(e, \mathcal P_i)\Big\}.
\]
We can now formulate the generalized Pyramidal Routing (\emph{genPR}) problem with instance $(G, c, W, T, b, i)$ as to minimize $\sum_{e \in E_G} c(e) y(e, \mathcal P_i)$ over all $\mathcal P_i$ (for an arbitrary fixed $i \in W$), where we define
\[
    y(e, \mathcal P_i) = \xi(e, \mathcal P_i) + \min\Big\{b(r),\ n(e, \mathcal P_i) - \xi(e, \mathcal P_i),\ k - n(e, \mathcal P_i) -  \xi(e, \mathcal P_i)\Big\}.
\]

We now formulate our version of Conjecture~2 (as in \cite{grandoni2008short}).
\renewcommand\theconjecture{2}
\begin{conjecture}[The \emph{genPR} conjecture]
    For each \emph{genPR} instance $(G, C, W, i, T, b)$ there exists an optimal solution which is a tree solution.
\end{conjecture}

We now show our version of Theorem~1 (having the same formulation as in \cite{grandoni2008short}), using the equivalent of Lemma~3, Claim~1, and Claim~2.

\renewcommand\thelemma{3}
\begin{lemma}
    Consider an generalized VPN instance $(G, c, W, T, b)$ with $T$ a two-union star with bridge $r$, $b(f) = 1$ for $f \neq r$, and some feasible solution $(\set{P_{ij}}, u)$.
    There exists a terminal $i \in W$ such that $\sum_{e \in E_G} c(e) u(e) \ge \sum_{e \in E_G} c(e) y(e, \mathcal P_i)$, where $\mathcal P_i = \set{P_{ij} : j \in W \setminus \set{i}}$.
\end{lemma}
\begin{proof}
    Fix an edge $e \in E_G$.
    We define the same traffic matrix $D^e = (d^e_{ij})_{i,j \in W}$ as in \cite{grandoni2008short},
    \[
        d^e_{ij} = \begin{cases}
                       \frac 1 k \left( \frac{y(e, \mathcal P_i)}{n(e, \mathcal P_i)} + \frac{y(e, \mathcal P_j)}{n(e, \mathcal P_j)} \right) & \text{if $e \in P_{ij}$,} \\
                       0 & \text{otherwise.}
        \end{cases}
    \]

    Now, the proof of Claim~1 is slightly different, as our universe is described by a different set of inequalities.

    \renewcommand\theclaim{1}
    \begin{claim}
        $D^e \in \mathcal U_T$, that is
        \[
            \sum_{\substack{ij \in \binom{W}{2}:\\f \in \pi_T(i,j)}} d^e_{ij} \le b(f)
        \]
        for all $f \in E_T$.
    \end{claim}
    \begin{proof}
        We consider two cases:
        \begin{itemize}
            \item $f \neq r$, hence we can write $f = f_i$, where $f_i \in E_T$ is the edge incident some terminal $i \in W$.
            Now note that we can write
            \[
                \sum_{\substack{\ell j \in \binom{W}{2}:\\f_i \in \pi_T(\ell,j)}} d^e_{\ell j} = \sum_{j \in W \setminus \set{i}} d^e_{ij}
            \]
            as $f_i$ is exactly in all paths from/to $i$ in the tree, as it is the edge incident to $i$.
            The rest of this case follows the same steps as the proof in \cite{grandoni2008short}:
            \[
                \begin{split}
                    \sum_{\substack{\ell j \in \binom{W}{2}:\\f_i \in \pi_T(\ell,j)}} d^e_{\ell j} &= \sum_{j \in W \setminus \set{i}} d^e_{ij} \\
                    &= \frac 1 k \sum_{\substack{j \in W \setminus \set{i}:\\ e \in P_{ij}}} \left( \frac{y(e, \mathcal P_i)}{n(e, \mathcal P_i)} + \frac{y(e, \mathcal P_j)}{n(e, \mathcal P_j)} \right) \\
                    &\le \frac 1 k \sum_{\substack{j \in W \setminus \set{i}:\\ e \in P_{ij}}} \left( \frac{k - n(e, \mathcal P_i)}{n(e, \mathcal P_i)} + \frac{n(e, \mathcal P_j)}{n(e, \mathcal P_j)} \right) \\
                    &= \frac{1}{n(e, \mathcal P_i)} \sum_{\substack{j \in W \setminus \set{i}:\\ e \in P_{ij}}} 1 \\
                    &= 1 \\
                    &= b(f_i).
                \end{split}
            \]

            \item $f = r$.
            We have not been able, with the current definition of $y$ and $D^e$, to make this work.
            However, we reduced it to a more accessible form.
            In the third line, we exchange the order of the two sums and use the symmetry of $\pi_T(i, j)$ and $P_{ij}$.
            \[
                \begin{split}
                    \sum_{\substack{ij \in \binom{W}{2},\\r \in \pi_T(i,j)}} d^e_{ij} &= \frac 1 2 \sum_{i \in W} \sum_{\substack{j \in W \setminus \set{i}:\\r \in \pi_T(i,j)}} d^e_{ij} \\
                    &= \frac{1}{2k} \sum_{i \in W} \sum_{\substack{j \in W \setminus \set{i}:\\r \in \pi_T(i,j),\\e \in P_{ij}}} \frac{y(e, \mathcal P_i)}{n(e, \mathcal P_i)} + \frac{1}{2k} \sum_{i \in W} \sum_{\substack{j \in W \setminus \set{i}:\\r \in \pi_T(i,j),\\e \in P_{ij}}} \frac{y(e, \mathcal P_j)}{n(e, \mathcal P_j)} \\
                    &= \frac{1}{2k} \sum_{i \in W} \sum_{\substack{j \in W \setminus \set{i}:\\r \in \pi_T(i,j),\\e \in P_{ij}}} \frac{y(e, \mathcal P_i)}{n(e, \mathcal P_i)} + \frac{1}{2k} \sum_{j \in W} \sum_{\substack{i \in W \setminus \set{j}:\\r \in \pi_T(j,i),\\e \in P_{ji}}} \frac{y(e, \mathcal P_j)}{n(e, \mathcal P_j)} \\
                    &= \frac{1}{k} \sum_{i \in W} \sum_{\substack{j \in W \setminus \set{i}:\\r \in \pi_T(i,j),\\e \in P_{ij}}} \frac{y(e, \mathcal P_i)}{n(e, \mathcal P_i)} \\
                    &= \frac{1}{k} \sum_{\substack{i \in W:\\n(e, \mathcal P_i) > 0}} \left( \alpha(e, \mathcal P_i) \frac{y(e, \mathcal P_i)}{n(e, \mathcal P_i)} \right) \\
                    &\ \vdots \text{\ (missing)}\\
                    &\le b(r).
                \end{split}
            \] \qedhere
        \end{itemize}

    \end{proof}
    The remainder of the proof of Lemma~3 (and Claim~2) is exactly the same as in \cite{grandoni2008short}, as the definition of $D^e$ is exactly the same, and the definition of $y(e, \mathcal P_i)$ itself is not used.
\end{proof}

From this follows (our version of) Theorem~1, stating that on ring networks, if there exists an optimal tree solution for the \emph{genPR} problem, that also an optimal tree solution exists for the generalized VPN problem.

Now, what remains to show is that indeed an optimal tree solution exists for the \emph{genPR} problem when $G$ is a ring.
If we follow the structure of the proof in \cite{grandoni2008short}, not first that Claim~3 also holds for our case.
However, Claim~4 does not hold: it might be that $y(e, \mathcal P_i) = y(f, \mathcal P_i) = b(r)$ (in the second case of the definition of $y$).

We are not sure that the counterexample for Claim~4 is actually problematic, or how to circumvent this issue in the last step of completing the argument.
