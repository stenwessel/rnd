\section{Generalized VPN problem}
To describe the generalized VPN problem, we first describe the class of tree demands.
Let $G = (V_G, E_G)$ be the underlying network graph with terminal set $W \subseteq V_G$.
Let an edge-capacitated tree $T = (V_T, E_T)$ be given with leaf set exactly the terminals $W$, and edge capacities $b_e$ for $e \in E_T$.
Then $T$ describes the tree demand universe $\mathcal U_T$ where a symmetric demand matrix $(D_{ij})$ belongs to $\mathcal U_T$ when it can be routed on $T$.
That is, for all edges $e$ on the (unique) path $\pi_T(i,j)$ between $i$ and $j$ in $T$ it must hold that $D_{ij} \le b_e$.
Note that if we take $T$ to be a star with center $r$ and edge capacities $b_{ir} = b_i$, then $\mathcal U_T$ describes the same universe as the hose model $\mathcal H$ with marginal demands $b_i$.
Hence, tree demands indeed generalize the hose model.

The generalized VPN is now fully characterized by the underlying network graph $G$, a terminal set $W$, edge per-unit-capacity costs $c_e$ ($e \in E_G$), and the \emph{demand tree} $T$ with edge capacities $b_f$ ($f \in E_T$).
A solution to the problem consists of a routing template $\mathcal P = \set{P_{ij} : i,j \in \binom W 2}$ and bought edge capacities $x_e$ for all $e \in E_G$.
We call a solution \emph{feasible} when all demand matrices in $\mathcal U_T$ can be routed on $G$ according to the routing template and without exceeding the installed edge capacities $x$.
We furthermore call a solution optimal if it is feasible and minimizes $\sum_{e \in E_G} c_e x_e$.

\subsection{Algorithms}

The generalized VPN problem can be solved with integer programming.
Each edge $uv$ has constant cost $c_{uv}$ and has a nonnegative variable $x_{uv}$ indicating the bought capacity for this edge.
The objective is then to minimize the total cost, i.e.\ the sum over all edges $uv$ of $c_{uv} x_{uv}$.

For each ordered pair of terminals $(i, j)$, a routing path between these terminals is constructed using binary flow variables.
The flow variable $f_{uv}^{ij}$ indicates whether the directed edge $(u, v)$ is used on the path from terminal $i$ to $j$.
This can be modeled with traditional flow constraints.
The routing path from $i$ to $j$ is then given by the set of directed edges $\set{(u, v) \in E_G | f_{uv}^{ij} = 1}$.
By symmetry, we only need to consider one ordering of all ordered pairs.
For ease of notation, we will still describe flow variables with $f_{uv}^{ij}$.

The last constraint ensures that we buy enough capacity on each edge.
For a demand matrix $D \in \mathcal U_T$, we can derive the required capacity on an edge $uv$ using the flow variables.
If $f^{ij}_{uv}$ or $f^{ij}_{vu}$ is 1, then we must buy $D_{ij}$ capacity on edge $uv$ to facilitate the flow between terminals $i$ and $j$.
Thus, we obtain the constraint
\[
    x_{uv} \ge \sum_{ij \in W \choose 2} D_{ij} ( f^{ij}_{uv} + f^{ij}_{vu}),
\]
which must hold for any edge $uv$ and for any demand matrix $D_{ij} \in \mathcal U_T$.
The complete program is defined in TODO ref. %TODO

This definition leads to an infinite number of constraints as $\mathcal U_T$ is not a finite set of matrices.
This issue can be resolved with row generation.
We only consider a subset $\mathcal U_T^* \subset \mathcal U_T$.
Then we obtain a solution that is valid for all $D \in \mathcal U_T^*$ and we try to find a matrix $D \in \mathcal U_T$ that violates the constraint.
If such a matrix exists, we add it to $\mathcal U_T^*$ and repeat until we cannot find any violating matrix.



\begin{alignat*}{5}
    \text{minimize}\ && \sum_{uv \in E_G} c_{uv} \cdot x_{uv} &&& \\
    \text{subject to}\ && x_{uv} &\ge \sum_{ij \in \binom{W}{2}} D_{ij} \cdot (f_{uv}^{ij} + f_{vu}^{ij}) &&\qquad \forall_{uv \in E_G,\ D \in \mathcal U_T} \\
    && \sum_{uv \in \delta(u)} (f_{uv}^{ij} - f_{vu}^{ij}) &= \begin{cases}
                                                                1 & \text{if $u = i$} \\
                                                                -1 & \text{if $u = j$} \\
                                                                0 & \text{otherwise}
    \end{cases} &&\qquad \forall_{u \in V_G,\ ij \in \binom{W}{2}} \\
    && x_{uv} &\in \mathbb{R}_+ &&\qquad \forall_{uv \in E_G} \\
    && f_{uv}^{ij},\ f_{vu}^{ij} &\in \{ 0, 1 \} &&\qquad \forall_{uv \in E_G,\ ij \in \binom{W}{2}}
\end{alignat*}%


Row generation subproblems (for all $uv \in E_G$)
            \begin{alignat*}{5}
                \text{maximize}\quad && \sum_{ij \in \binom{W}{2}} (\tilde f_{uv}^{ij} + \tilde f_{vu}^{ij}) \cdot D_{ij} &&& \\
                \text{subject to}\quad && \sum_{\substack{ij \in \binom{W}{2}\\e \in \pi_T(i,j)}} D_{ij} &\le b_e &&\qquad \forall_{e \in E_T} \\
                && D_{ij} &\in \mathbb{R}_+ &&\qquad \forall_{ij \in \binom{W}{2}}
            \end{alignat*}
            If $D^*$ has objective $> \tilde x_{uv}$, add $D^*$ to $\mathcal U_T^*$.

\begin{alignat*}{5}
    \text{maximize}\ && \sum_{ij \in \binom{W}{2}} (f_{uv}^{ij} + f_{vu}^{ij}) \cdot D_{ij} &&& \\
    \text{subject to}\ && \sum_{\substack{ij \in \binom{W}{2}\\e \in \pi_T(i,j)}} D_{ij} &\le b_e &&\qquad \forall_{e \in E_T} \qquad (\omega_e^{uv}) \\
    && D_{ij} &\in \mathbb{R}_+ &&\qquad \forall_{ij \in \binom{W}{2}}
\end{alignat*}
\hrulefill
\begin{alignat*}{5}
    \text{minimize}\ && \sum_{e \in E_T} b_e \cdot \omega_e^{uv} &&& \\
    \text{subject to}\ && \sum_{e \in \pi_T(i,j)} \omega_e^{uv} &\ge f_{uv}^{ij} + f_{vu}^{ij} &&\qquad \forall_{ij \in \binom{W}{2}} \\
    && \omega_e^{uv} &\in \mathbb{R}_+ &&\qquad \forall_{e \in E_T}
\end{alignat*}

\begin{alignat*}{5}
    \text{minimize}\ && \sum_{uv \in E_G} c_{uv} \cdot x_{uv} &&& \\
    \text{subject to}\ && x_{uv} &\ge \sum_{e \in E_T} b_e \cdot \omega_e^{uv} &&\qquad \forall_{uv \in E_G} \\
    && \sum_{e \in \pi_T(i,j)} \omega_e^{uv} &\ge f_{uv}^{ij} + f_{vu}^{ij} &&\qquad \forall_{uv \in E_G,\ ij \in \binom{W}{2}} \\
    && \sum_{uv \in \delta(u)} (f_{uv}^{ij} - f_{vu}^{ij}) &= \begin{cases}
                                                                1 & \text{if $u = i$} \\
                                                                -1 & \text{if $u = j$} \\
                                                                0 & \text{otherwise}
    \end{cases} &&\qquad \forall_{u \in V_G,\ ij \in \binom{W}{2}} \\
    && x_{uv} &\in \mathbb{R}_+ &&\qquad \forall_{uv \in E_G} \\
    && \omega_e^{uv} &\in \mathbb{R}_+ &&\qquad \forall_{uv \in E_G,\ e \in E_T} \\
    && f_{uv}^{ij},\ f_{vu}^{ij} &\in \{ 0, 1 \} &&\qquad \forall_{uv \in E_G,\ ij \in \binom{W}{2}}
\end{alignat*}%

\begin{algorithm}
    \caption{RecursiveHubbing}
    \label{alg:TriangulateStar}
    \begin{algorithmic}[1]
        \Statex Recursive enumeration algorithm to find all optimal mappings. Initially, verticesToAssign equals $V_T - W$ and $h$ is an empty mapping from $V_T$ to $V_G$. In the actual implementation, we also maintain a set of all best mappings that attain the minimum cost.

        \Procedure{Assign}{verticesToAssign, $h$}
            \If {verticesToAssign is empty}
                \State \Return $\sum_{\set{u, v} \in E_T} d(h(u), h(v))$ \Comment{The cost of mapping $h$}
            \EndIf

            \State $u$ = verticesToAssign.first()
            \State bestCost = $\infty$

            \ForAll {vertices $v$ in graph}
                \State $h(u)$ = $v$
                \State currentCost = \textsc{Assign}(verticesToAssign $-$ $u$, $h$)

                \If {currentCost $<$ bestCost}
                    \State bestCost = currentCost
                \EndIf
            \EndFor

            \State \Return bestCost
        \EndProcedure
    \end{algorithmic}
\end{algorithm}

%TODO Die recurrence een reference geven (hahaha sorry alvast)
\begin{algorithm}
    \caption{Finding all subtrees for the dynamic programming algorithm}
    \label{alg:idk}
    \begin{algorithmic}[1]
        \Statex Obtaining all subtrees that are needed for the dynamic programming algorithm.
        The dynamic program itself only fills in the table based on the recurrence stated below.
        \Procedure{GetSubtrees}{$V_T, E_T, W$}
            \State root = arbitrary node in $V_T - W$

            \State $\mathcal S$ = empty list
            \State $\mathcal S$.add(Subtree(root, root.neighbors, null)) \Comment{A subtree has a root, children and parent}
            \State $\mathcal Q$.addAll($\mathcal S$)

            \While {$\mathcal Q$ is not empty}
                \State tree = $\mathcal Q$.removeFirst
                \State currentRoot = tree.root
                \ForAll{vertices $r$ in tree.children}
                    \State $S$ = Subtree($r$, $r$.neighbors $-$ currentRoot, currentRoot)
                    \State $\mathcal S$.add($S$)
                    \State $\mathcal Q$.add($S$)
                \EndFor
            \EndWhile
            \State \Return subtrees.reversed()
        \EndProcedure
    \end{algorithmic}
\end{algorithm}

Recurrence:
\[
    T[S, v] = \begin{cases}
                  0 &\text{if $S$.root $\in W$ and $S$.root = $v$} \\
                  \infty &\text{if $S$.root $\in W$ and $S$.root $\neq$ $v$} \\
                  \displaystyle \sum_{r \in S.\text{children}} \min_{w \in V_G} \Big( T[S_r, w] + b(\set{r, S.\text{root}}) \cdot d(v, w) \Big) &\text{otherwise} \\
    \end{cases}
\]
When the table is completely filled, the returned result is obtained by computing $\min_{v \in V_G} T[S_T, v]$, where $S_T$ denotes the (unique) subtree containing all vertices of $T$.

\subsection{Experiments}
All the experiments we have seen.
Integrality gap results.

\subsection{Ring networks}
The paper by Grandoni et al.~\cite{grandoni2008short} gives a short proof that the VPN conjecture is true for the cases where the considered network is a ring.
In the following, we will discuss whether it is possible to extend the proof to the generalized VPN conjecture, for the cases where the network is a ring and the demand tree is a \emph{two-union star}: the tree formed by connecting the centers of two star graphs.

Some background of what steps we are taking is identical to what is described in \cite{grandoni2008short} and we will omit these details.
We focus on showing the differences that are necessary in the proof to adapt to the generalized VPN problem setting.
We use the same notation as used in the aforementioned paper, except where the generalized VPN problem differs from the VPN problem.

For the remainder, we restrict ourselves to instances where $T$ is a \emph{two-union star}, that is, a tree formed by connecting the centers $r_1,\ r_2$ of two star graphs.
We will call this edge $r = \set{r_1, r_2}$ and refer to it as the \emph{bridge} of $T$.

\subsubsection{Preliminaries}
A number of assumptions are made in \cite{grandoni2008short}, the most important being that we can assume the capacities $b \equiv 1$.
We suspect this does not quite generalize to our case, but we can state the following:

\begin{fact}
    We may assume $b(f) = 1$ when $f \neq r$.
\end{fact}
Using the same motivation as in \cite{grandoni2008short}, if the capacity $b(iu)$ for the terminal $i \in W$ is not unit, construct a new instance with new terminals $i_1, \dots, i_{b(iu)}$, connected to the site of $i$ in $G$, with edge cost $0$.
Set the capacities of the incident edges in $T$ to unit.
Note that this instance is equivalent, but may change the topology of the graph.

An alternative construction is argued in \cite{grandoni2008short} to keep the topology a ring and we refer to this paper for the motivation.

\begin{fact}
    We can assume $b$ to be integral.
\end{fact}
This is motivated by scaling by a sufficiently large factor (excluding weird irrational cases, but that is probably fine).

\subsubsection{Pyramidal Routing problem}
To goal is to show that there exists an optimal solution to the generalized VPN problem that is a tree solution (or, equivalently, there exists a tree solution to the generalized VPN problem that is optimal).
Consider an optimal tree solution $(\set{P_{ij}}, u)$ to a generalized VPN instance $(G, c, W, T, b)$, with $|W| = k$ and $T$ a two-union star with the unit capacity assumption as above.
Let $\mathcal P_i$ be as in \cite{grandoni2008short} for a fixed $i \in W$.
We define
\[
    \xi(e, \mathcal P_i) \coloneqq \min\set{\alpha(e, \mathcal P_i),\ \beta(e, \mathcal P_i)} + \min\set{n(e, \mathcal P_i) - \alpha(e, \mathcal P_i),\ k - n(e, \mathcal P_i) - \beta(e, \mathcal P_i)}
\]
where
\begin{gather*}
    n(e, \mathcal P_i) \coloneqq |\set{j \in W \setminus \set{i} : e \in P_{ij}}|,\\
    \alpha(e, \mathcal P_i) \coloneqq |\set{j \in W \setminus \set{i} : e \in P_{ij},\ r \in \pi_T(i, j)}|,\\
    \beta(e, \mathcal P_i) \coloneqq |\set{j \in W \setminus \set{i} : e \not\in P_{ij},\ r \not\in \pi_T(i, j)}|,
\end{gather*}
that is, $\alpha(e, \mathcal P_i)$ is the number of paths in $\mathcal P_i$ containing $e$, \emph{while simultaneously the path in the demand tree between $i$ and $j$ crosses the bridge}.
In our setting, we have a different expression for the \emph{required} capacity of edge $e$:
\[
    u(e) = \xi(e, \mathcal P_i) + \min\Big\{b(r),\ n(e, \mathcal P_i) - \xi(e, \mathcal P_i),\ k - n(e, \mathcal P_i) -  \xi(e, \mathcal P_i)\Big\}.
\]
We can now formulate the generalized Pyramidal Routing (\emph{genPR}) problem with instance $(G, c, W, T, b, i)$ as to minimize $\sum_{e \in E_G} c(e) y(e, \mathcal P_i)$ over all $\mathcal P_i$ (for an arbitrary fixed $i \in W$), where we define
\[
    y(e, \mathcal P_i) = \xi(e, \mathcal P_i) + \min\Big\{b(r),\ n(e, \mathcal P_i) - \xi(e, \mathcal P_i),\ k - n(e, \mathcal P_i) -  \xi(e, \mathcal P_i)\Big\}.
\]

We now formulate our version of Conjecture~2 (as in \cite{grandoni2008short}).
\renewcommand\theconjecture{2}
\begin{conjecture}[The \emph{genPR} conjecture]
    For each \emph{genPR} instance $(G, C, W, i, T, b)$ there exists an optimal solution which is a tree solution.
\end{conjecture}

We now show our version of Theorem~1 (having the same formulation as in \cite{grandoni2008short}), using the equivalent of Lemma~3, Claim~1, and Claim~2.

\renewcommand\thelemma{3}
\begin{lemma}
    Consider an generalized VPN instance $(G, c, W, T, b)$ with $T$ a two-union star with bridge $r$, $b(f) = 1$ for $f \neq r$, and some feasible solution $(\set{P_{ij}}, u)$.
    There exists a terminal $i \in W$ such that $\sum_{e \in E_G} c(e) u(e) \ge \sum_{e \in E_G} c(e) y(e, \mathcal P_i)$, where $\mathcal P_i = \set{P_{ij} : j \in W \setminus \set{i}}$.
\end{lemma}
\begin{proof}
    Fix an edge $e \in E_G$.
    We define the same traffic matrix $D^e = (d^e_{ij})_{i,j \in W}$ as in \cite{grandoni2008short},
    \[
        d^e_{ij} = \begin{cases}
                       \frac 1 k \left( \frac{y(e, \mathcal P_i)}{n(e, \mathcal P_i)} + \frac{y(e, \mathcal P_j)}{n(e, \mathcal P_j)} \right) & \text{if $e \in P_{ij}$,} \\
                       0 & \text{otherwise.}
        \end{cases}
    \]

    Now, the proof of Claim~1 is slightly different, as our universe is described by a different set of inequalities.

    \renewcommand\theclaim{1}
    \begin{claim}
        $D^e \in \mathcal U_T$, that is
        \[
            \sum_{\substack{ij \in \binom{W}{2}:\\f \in \pi_T(i,j)}} d^e_{ij} \le b(f)
        \]
        for all $f \in E_T$.
    \end{claim}
    \begin{proof}
        We consider two cases:
        \begin{itemize}
            \item $f \neq r$, hence we can write $f = f_i$, where $f_i \in E_T$ is the edge incident some terminal $i \in W$.
            Now note that we can write
            \[
                \sum_{\substack{\ell j \in \binom{W}{2}:\\f_i \in \pi_T(\ell,j)}} d^e_{\ell j} = \sum_{j \in W \setminus \set{i}} d^e_{ij}
            \]
            as $f_i$ is exactly in all paths from/to $i$ in the tree, as it is the edge incident to $i$.
            The rest of this case follows the same steps as the proof in \cite{grandoni2008short}:
            \[
                \begin{split}
                    \sum_{\substack{\ell j \in \binom{W}{2}:\\f_i \in \pi_T(\ell,j)}} d^e_{\ell j} &= \sum_{j \in W \setminus \set{i}} d^e_{ij} \\
                    &= \frac 1 k \sum_{\substack{j \in W \setminus \set{i}:\\ e \in P_{ij}}} \left( \frac{y(e, \mathcal P_i)}{n(e, \mathcal P_i)} + \frac{y(e, \mathcal P_j)}{n(e, \mathcal P_j)} \right) \\
                    &\le \frac 1 k \sum_{\substack{j \in W \setminus \set{i}:\\ e \in P_{ij}}} \left( \frac{k - n(e, \mathcal P_i)}{n(e, \mathcal P_i)} + \frac{n(e, \mathcal P_j)}{n(e, \mathcal P_j)} \right) \\
                    &= \frac{1}{n(e, \mathcal P_i)} \sum_{\substack{j \in W \setminus \set{i}:\\ e \in P_{ij}}} 1 \\
                    &= 1 \\
                    &= b(f_i).
                \end{split}
            \]

            \item $f = r$.
            We have not been able, with the current definition of $y$ and $D^e$, to make this work.
            However, we reduced it to a more accessible form.
            In the third line, we exchange the order of the two sums and use the symmetry of $\pi_T(i, j)$ and $P_{ij}$.
            \[
                \begin{split}
                    \sum_{\substack{ij \in \binom{W}{2},\\r \in \pi_T(i,j)}} d^e_{ij} &= \frac 1 2 \sum_{i \in W} \sum_{\substack{j \in W \setminus \set{i}:\\r \in \pi_T(i,j)}} d^e_{ij} \\
                    &= \frac{1}{2k} \sum_{i \in W} \sum_{\substack{j \in W \setminus \set{i}:\\r \in \pi_T(i,j),\\e \in P_{ij}}} \frac{y(e, \mathcal P_i)}{n(e, \mathcal P_i)} + \frac{1}{2k} \sum_{i \in W} \sum_{\substack{j \in W \setminus \set{i}:\\r \in \pi_T(i,j),\\e \in P_{ij}}} \frac{y(e, \mathcal P_j)}{n(e, \mathcal P_j)} \\
                    &= \frac{1}{2k} \sum_{i \in W} \sum_{\substack{j \in W \setminus \set{i}:\\r \in \pi_T(i,j),\\e \in P_{ij}}} \frac{y(e, \mathcal P_i)}{n(e, \mathcal P_i)} + \frac{1}{2k} \sum_{j \in W} \sum_{\substack{i \in W \setminus \set{j}:\\r \in \pi_T(j,i),\\e \in P_{ji}}} \frac{y(e, \mathcal P_j)}{n(e, \mathcal P_j)} \\
                    &= \frac{1}{k} \sum_{i \in W} \sum_{\substack{j \in W \setminus \set{i}:\\r \in \pi_T(i,j),\\e \in P_{ij}}} \frac{y(e, \mathcal P_i)}{n(e, \mathcal P_i)} \\
                    &= \frac{1}{k} \sum_{\substack{i \in W:\\n(e, \mathcal P_i) > 0}} \left( \alpha(e, \mathcal P_i) \frac{y(e, \mathcal P_i)}{n(e, \mathcal P_i)} \right) \\
                    &\ \vdots\\
                    &\le b(r).
                \end{split}
            \] \qedhere
        \end{itemize}

    \end{proof}
    The remainder of the proof of Lemma~3 (and Claim~2) is exactly the same as in \cite{grandoni2008short}, as the definition of $D^e$ is exactly the same, and the definition of $y(e, \mathcal P_i)$ itself is not used.
\end{proof}

From this follows (our version of) Theorem~1, stating that on ring networks, if there exists an optimal tree solution for the \emph{genPR} problem, that also an optimal tree solution exists for the generalized VPN problem.

Now, what remains to show is that indeed an optimal tree solution exists for the \emph{genPR} problem when $G$ is a ring.
If we follow the structure of the proof in \cite{grandoni2008short}, not first that Claim~3 also holds for our case.
However, Claim~4 does not hold: it might be that $y(e, \mathcal P_i) = y(f, \mathcal P_i) = b(r)$ (in the second case of the definition of $y$).
We have not quite yet mastered the inductive proof that follows (in which Claim~4) is used.
Therefore, we are not sure that the counterexample for Claim~4 is actually problematic, or how to circumvent this issue.
