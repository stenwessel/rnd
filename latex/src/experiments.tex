%! Author = Sten
%! Date = 09-10-2020

\documentclass[11pt]{article}

\usepackage[a4paper,margin=2.5cm]{geometry}

% Packages
\usepackage[fleqn]{amsmath}
\usepackage{hyperref}
\usepackage{graphicx}

\usepackage{amsthm}
\usepackage{amssymb}
\usepackage{stmaryrd}
\usepackage{amsfonts}
\usepackage{etoolbox}
\usepackage{enumerate}
\usepackage{scalerel}
\usepackage{stackengine}
\usepackage{xifthen}
\usepackage{lmodern}
\usepackage{gensymb}
\usepackage{color}
\usepackage{colortbl}
\usepackage{needspace}
\usepackage{enumitem}
\usepackage{mathtools}
\usepackage{bbm}
\usepackage{xargs}
\usepackage{algorithm}
\usepackage{diagbox}
\usepackage{subcaption}
\usepackage[table]{xcolor}
\usepackage[noend]{algpseudocode}
\usepackage{polynom}

\newcommand{\naturals}{\mathbb{N}}
\newcommand{\integers}{\mathbb{Z}}
\newcommand{\reals}{\mathbb{R}}
\newcommand{\complex}{\mathbb{C}}
\newcommand{\rationals}{\mathbb{Q}}

\DeclarePairedDelimiterX{\set}[1]\lbrace\rbrace{\setaux #1||\endsetaux}
\def\setaux#1|#2|#3\endsetaux{\if\relax\detokenize{#2}\relax #1 \else #1 \;\delimsize\vert\; #2 \fi}

% Q.E.D.
\renewcommand{\qedsymbol}{\footnotesize$\boxempty$ \textsc{q.\,e.\,d.}}


\title{\textbf{Generalized VPN project}\\Summary of experiments}
\author{David Dekker \and Sten Wessel}
\date{\today}

% Document
\begin{document}
    \maketitle
    \hrule


    \section{Generalized VPN}
    Implemented algorithms:
    \begin{itemize}
        \item Enumeration of all hubbing solutions
        \item Dynamic programming for finding the optimal hubbing solution
        \item Compact MIP for the optimal solution
    \end{itemize}

    \noindent%
    Experiments:
    \begin{itemize}
        \item Petersen graph (unit edge cost) with 2-union star demand tree.

        For each possible subset of vertices of Petersen that we mark as terminal:
        we have tried every possible `partition' of terminals on left/right of the bridge of the demand tree, with all relevant integer choices of the bridge capacity.

        These are $68\,185$ instances in total.
        No counterexamples.
        No integrality gap for all instances.
        Essentially nothing interesting.

        \item
        Petersen graph with random edge costs, random terminal selection, randomly built demand tree (not restricting to 2-union stars).
        For more than $100\,000$ instances, no interesting results.

        \item Augmented Petersen graph: Petersen graph with unit edge costs, with additional `root' vertex, connected to all Petersen vertices with edge cost 2.

        Demand tree:

        \includegraphics[width=.2\textwidth]{petdt.png}

        We tried various capacities in the tree (red numbers in the figure).
        Groups of 3 are defined by, for each edge $uv$, choose an endpoint (say $v$), create the group with $N[v] \setminus \{ u \}$.
        This will create 15 groups.
        Choice of endpoint is random.
        For each combination of capacities in the demand tree, we tried about 200 different sets of groups;
        none yielded a counterexample.

        We also tried adding two groups for each edge (one for each endpoint), yielding 30 groups total in the demand tree (and---surprisingly---no counterexample).

        \item Realistic VPN instances, with randomly generated demand trees with capacities based on the terminal capacities from the VPN instance (capacities of the `internal' edges of the demand tree are randomly sampled from the normal distribution constructed from the sample mean/sample variance of the terminal capacities in the VPN instance).
        Most instances are (very) large, and the MIP takes an enormous amount of time and memory to solve these, so we only have a few results, no counterexamples found.
        However, we don't expect to find any counterexample here.

        \item `Integrality gap' instances from the ACM paper ``The VPN Conjecture is True''.
        We interpolate the figures there to a class of graphs with $k$ terminals (examples for $k=4$ and $k=6$ below).
        We assume unit costs on the edges.

        For the regular VPN problem (i.e.\ the demand tree is a star), there is an integrality gap.
        We try to see whether the integrality gap also exists for a 2-union star demand tree.
        Note that some choices of 2-union stars are equivalent to a normal star (e.g.\ a sufficiently large bridge capacity, or when there is at most one terminal on a side), so we only look at instances really different from the regular VPN case.

        We do see some 2-union stars where there is no integrality gap (but where the regular VPN equivalent has one).
        We thought up until yesterday (Wednesday) that we could characterize for which 2-union stars we see this behavior, but it turns out that non-integer capacities of the bridge invalidate this characterization.
        We hope to give more details tomorrow in the meeting.

    \end{itemize}

    \section{Capped hose model}
    Implemented algorithms:
    \begin{itemize}
        \item Dynamic programming for finding the optimal hubbing solution
        \item Compact MIP for the optimal solution
    \end{itemize}

    So far we have no real experiments other than some toy examples to confirm our initial implementation was incorrect (we believe our current version is correct).
\end{document}
